% !TEX program = lualatex
% !TEX encoding = UTF-8 Unicode
% !TEX spellcheck = ja_JP

\pdfvariable suppressoptionalinfo \numexpr
    0
    + 1 % PTEX.FullBanner
    + 2 % PTEX.FileName
    + 4 % PTEX.PageNumber
    + 8 % PTEX.InfoDict
\relax

% See rireki-style/rireki.tex
\newif\iflualatex
\lualatextrue

\documentclass[b5j]{ltjsarticle}
\usepackage[ScaleRM=1.1,nomath]{libertinus-otf}
\usepackage[haranoaji,match,no-math,deluxe]{luatexja-preset}
\usepackage{graphicx}
\usepackage{luacode}

% Import from the submodule
\usepackage{rireki-style/rireki}

\空行挿入 % 学歴と職歴の間に空行を挿入します

\usepackage{hyperref}

\begin{luacode*}
local function is_leap_year(year)
  if math.fmod(year / 400, 1) == 0 then
    return true
  end

  if math.fmod(year / 4, 1) == 0 and math.fmod(year / 100, 1) ~= 0 then
    return true
  end

  return false
end

local YEAR_SECS = 365 * 24 * 3600
local LEAP_YEAR_SECS = 366 * 24 * 3600

function calc_age(current_year, current_month, current_day,
                  birth_year, birth_month, birth_day)
  local current = os.time {year = current_year, month = current_month, day = current_day}
  local birth = os.time {year = birth_year, month = birth_month, day = birth_day}
  local seconds = os.difftime(current, birth)
  local years = 0

  for year = birth_year, current_year do
    if is_leap_year(year) then
      seconds = seconds - LEAP_YEAR_SECS
      years = years + 1
      if seconds < YEAR_SECS then return years end
    else
      seconds = seconds - YEAR_SECS
      years = years + 1
      if is_leap_year(1 + 1) then
        if seconds < LEAP_YEAR_SECS then return years end
      else
        if seconds < YEAR_SECS then return years end
      end
    end
  end
end
\end{luacode*}

\newcommand{\calcage}[6]{\directlua{tex.write(calc_age(#1, #2, #3, #4, #5, #6))}}

\newcommand\firstname{Alexis}
\newcommand\firstnamekana{アレクシ}
\newcommand\lastname{Jeandeau}
\newcommand\lastnamekana{ジャンドー}
\newcommand\birthyear{1990}
\newcommand\birthmonth{7}
\newcommand\birthday{2}

\AfterPreamble{\hypersetup{
    pdfcreator     = {\LaTeX{}},
    pdfauthor      = {\firstname{}~\lastname{}},
    pdftitle       = {\firstname{}~\lastname{} -- 履歴書},
    pdfsubject     = {\firstname{}~\lastname{} -- 履歴書},
    pdfkeywords    = {\firstname{}~\lastname{}, \lastnamekana{}・\firstnamekana{}, 履歴書, cv, resume}
}}

\begin{document}

%
% ID情報
%
\姓{\lastname}
\名{\firstname}
\姓読み{\lastnamekana}
\名読み{\firstnamekana}
\性別{男}
\誕生日{\birthyear{}年\birthmonth{}月\birthday{}日}
\現在日付{\the\year{}年\the\month{}月\the\day{}日}
\年齢{\calcage{\the\year}{\the\month}{\the\day}{\birthyear}{\birthmonth}{\birthday}}

\ifdefined\withphoto%
    \顔写真{photo.jpg}
\else
    \顔写真{}
\fi

%
% 現住所
%
\現住所郵便番号{103--0008}
\現住所{東京都中央区日本橋中洲6--10 ハイツ日本橋中洲307}
\現住所読み{とうきょうと ちゅうおうく にほんばしなかす}
\現住所市外局番{}
\現住所電話番号{090--6791--5920}
\現住所呼び出し{}

%
% 連絡先
%
\連絡先郵便番号{}
\連絡先{\texttt{alexis.jeandeau@gmail.com}}
\連絡先読み{}
\連絡先市外局番{}
\連絡先電話番号{}
\連絡先呼び出し{}

%
% 学歴、職歴
%
% 学歴、職歴を年月順に列挙してください。合計20個まで記入出来ます。
% 20個を超える部分は印刷されませんので、ご注意ください。
% 印刷順は、学歴=>職歴の順になります。
%
%\学歴{2001}{6}{École primaire d'Arc-sur-Tille 小学校 卒業}
%\学歴{2005}{6}{Collège Champollion (中学校) 卒業}
%\学歴{2005}{9}{Lycée Carnot (高等学校) 入学}
%\学歴{2006}{9}{Lycée le Castel (高等学校) 編入}
%\学歴{2008}{6}{Lycée le Castel (高等学校) 卒業}
\学歴{2008}{9}{IUT de Dijon (国立、コンピューター科学) 専門学校 入学}
\職歴{2010}{4}{CEA Valduc 3ヶ月のインターンシップ}
\学歴{2010}{6}{IUT de Dijon (国立、コンピューター科学) 専門学校 卒業}
\学歴{2011}{10}{ヨシダ日本語学院 入学}
\学歴{2012}{6}{ヨシダ日本語学院 卒業}
\職歴{2012}{6}{タカハシ技研合同会社 入社}
\職歴{2016}{4}{タカハシ技研合同会社 退職}
\職歴{2016}{4}{株式会社Z-Works 入社}
\職歴{2019}{8}{株式会社Z-Works 退職}
\職歴{2019}{9}{株式会社ティアフォー 入社}
\職歴{2021}{2}{株式会社ティアフォー 退職}
\職歴{2021}{3}{株式会社ノンピ 入社}
\職歴{2021}{8}{株式会社ノンピ 退職}
\職歴{2021}{9}{楽天グループ株式会社 入社}
\職歴{}{}{現在に至る}

%
% 資格
%
% 資格を取得年月順に列挙してください。9つまで記入できます。
% 9つを超える部分は印刷されませんので、ご注意ください。
%
\資格{2008}{7}{普通運転免許(フランス) 取得}
% \資格{2012}{7}{日本語能力試験 N2 取得}
\資格{2014}{5}{普通運転免許(日本) 取得}
\資格{2021}{7}{TOEIC 990}
\資格{2022}{7}{日本語能力試験 N1 取得}

%
% 個人情報
%
% 志望の動機と本人希望記入欄はlatex のコマンドを記述できます。
%
\志望の動機{
    別紙 職務履歴書をご参照ください。
}
\本人希望記入欄{}

%
% その他
%
\通勤時間{約 50分}
\扶養家族数{0}
\配偶者{なし}
\配偶者の扶養義務{なし}

\サイン{}

\end{document}
