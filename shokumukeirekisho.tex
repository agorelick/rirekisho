% !TEX program = lualatex

\documentclass[11pt,a4paper]{moderncv}

% moderncv theme
% Style options are 'casual' (default), 'classic', 'oldstyle' and 'banking'
\moderncvstyle{classic}
% Color options 'blue' (default), 'orange', 'green', 'red', 'purple', 'grey'
% and 'black'
\moderncvcolor{black}

\usepackage{xpatch}
% Patch the \cventry command so that it doesn't use italics
\xpatchcmd{\cventry}{\slshape#4}{#4}{}{}
% Remove the comma in between the job title / degree
% and the institution / employer
\xpatchcmd\cventry{,}{}{}{}
% Remove the period after the institution / employer
\xpatchcmd\cventry{.}{}{}{}

\usepackage[deluxe,nfssonly]{luatexja-preset}

% Adjust the page margins
\usepackage[scale=0.9]{geometry}
\usepackage{microtype}

\usepackage{fontspec}
\usepackage{libertinus}
\usepackage{polyglossia}
\setdefaultlanguage{english}
\setotherlanguage{french}

\usepackage[CJK,overlap]{ruby}
\usepackage{setspace}
% See https://nablux.net/tgp/weblog/2013/03/22/how-typeset-japanese-using-xelatex/
% Increase line space for Furigana
\onehalfspacing%

% Redefine the date of today in Japanese, in order to display
% XXXX年XX月XX日現在on top of the first page
\renewcommand{\today}{\the\year{}年\the\month{}月\the\day{}日}

% Default for the "classic" style: marvosym
% "awesome" is the default in the "banking", "casual" and "classic" styles
\moderncvicons{awesome}

% Personal data
\lineskiplimit=5pt
\name{\ruby{Jeandeau}{ジャンドー}}{\ruby{Alexis}{アレクシ}}
\lineskiplimit=1pt
\title{職務経歴書}

\ifdefined\withphoto%
    % '64pt' is the height the picture must be resized to,
    % 0.4pt is the thickness of the frame around it (put it to 0pt for no frame)
    % and 'picture' is the name of the picture file
    \photo[64pt][0.2pt]{photo.jpg}
\fi

% Technologies / protocols
\providecommand{\techno}{}
\renewcommand{\techno}{\emph}
% Languages / libraries
\providecommand{\lang}{}
\renewcommand{\lang}{\emph}

\AfterPreamble{\hypersetup{
    % pdfcreator     = {\LaTeX{} with 'moderncv' package},
    pdfcreator     = {\LaTeX{}},
    pdfauthor      = {Alexis~Jeandeau},
    pdftitle       = {Alexis~Jeandeau -- 職務経歴書},
    pdfsubject     = {Alexis~Jeandeau -- 職務経歴書},
    pdfkeywords    = {Alexis~Jeandeau, 職務経歴書}
}}

%
% Content
%

\begin{document}

\makecvtitle%

\section{職務経歴}

\subsection{\underline{2019年9月〜 株式会社ティアフォー}}

\cventry{}{}{•技術本部所属 Webエンジニア}{}{}{%
\begin{itemize}
\item[] 【プロジェクト内容】
    \begin{itemize}
        \item 自動運転車両を運行管理するMaaSのデータ基盤の開発
        \item 自動運転車両とクラウドを繋ぐメッセージング機能開発
        \item 自動運転システムの状態・バージョン管理機能の開発
    \end{itemize}
\item[] 【技術】
    \begin{description}
        \item \techno{\href{https://aws.amazon.com/iot/}{AWS IoT}}, \techno{MQTT}, \techno{Python}, \techno{C++}
    \end{description}
\end{itemize}
}

\vspace*{1em}

\subsection{\underline{2016年4月〜2019年8月 株式会社Z-Works}}

\cventry{}{}{•介護者むけのIoTプロットフォームのバックエンド開発}{}{}{%
\begin{itemize}
\item[] 【プロジェクト内容】
    \begin{itemize}
        \item ErlangやJavaを使ってIoTのデバイスとユーザを管理するシステムの開発
        \item AMQP(RabbitMQ)通信APIの実装
        \item Bitbucket Pipelines、AWS CodeDeployと複数のAmazonサービスでCI/CDの設定
    \end{itemize}
\item[] 【技術】
    \begin{description}
        \item \lang{Erlang}, \lang{Java}, \techno{AMQP}, \techno{Z-Wave}, \techno{\href{http://www.linear.com/products/smartmesh_ip}{Dust Networks SmartMesh IP}}, \techno{PostgreSQL}
    \end{description}
\end{itemize}
}
\cventry{}{}{•サポートエンジニア向けの管理インターフェスの開発}{}{}{%
\begin{itemize}
\item[] 【プロジェクト内容】
    \begin{itemize}
        \item AWSのサービスを使ってREST APIの作成
    \end{itemize}
\item[] 【技術】
    \begin{description}
        \item \lang{Java}, \href{https://aws.amazon.com/api-gateway/}{Amazon API Gateway}, \href{https://aws.amazon.com/lambda/}{AWS Lambda}
    \end{description}
\end{itemize}
}

\vspace*{1em}

\subsection{\underline{2012年5月〜2016年4月 タカハシ技研合同会社}}

\cventry{}{}{•イベント用のリアルタイム写真閲覧と管理サービス}{}{}{%
\begin{itemize}
\item[] 【プロジェクト内容】
    \begin{itemize}
        \item バックエンドとREST APIの設計
        \item フロントエンドの国際化(日本語・英語)
    \end{itemize}
\item[] 【技術】
    \begin{description}
        \item \lang{Go}, \lang{JavaScript}, \techno{WebSocket}, \techno{PostgreSQL}, \techno{L20n}
    \end{description}
\end{itemize}
}
\vspace*{1em}
\cventry{}{}{•LTE-A (4G) のネットワークテストツールの開発}{}{}{%
\begin{itemize}
\item[] 【プロジェクト内容】
    \begin{itemize}
        \item 複数のネットワークプロトコルの負荷テストテストモジュールの作成\newline{(HTTP, FTP, ICMP, SIP, RTP)}
    \end{itemize}
\item[] 【技術】
    \begin{description}
        \item \lang{C++}, \techno{Boost}
    \end{description}
\end{itemize}
}
\vspace*{1em}
\cventry{}{}{•フェリカ(FeliCa)カードを利用する勤怠管理システムの開発}{}{}{%
\begin{itemize}
\item[] 【プロジェクト内容】
    \begin{itemize}
        \item 出勤簿閲覧などのウェッブインターフェースの開発
        \item フロントエンドの国際化(日本語・英語・フランス語)
    \end{itemize}
\item[] 【技術】
    \begin{description}
        \item \lang{Python}, \lang{JavaScript}, \techno{PostgreSQL}, \techno{NFC}
    \end{description}
\end{itemize}
}

% Increase the length of the left column (used for the employment dates)
\setlength{\hintscolumnwidth}{2.3cm}

\section{言語}
\cvitemwithcomment{日本語}{中級者 -- 2012年に日本語能力試験2級を取得}{}
\cvitemwithcomment{英語}{上級者}{}
\cvitemwithcomment{フランス語}{母語}{}

\vspace*{1em}

\section{スキル}
\subsection{プログラミング言語}
\cvitem{一般}{\lang{C}, \lang{C++}, \lang{Java}}
\cvitem{サーバ}{\lang{Erlang}, \lang{Python}, \lang{Ruby}, \lang{Go}, \lang{Node.js}}
\subsection{技術}
\cvitem{OS}{\techno{RHEL}, \techno{CentOS}, \techno{Debian}, \techno{Ubuntu}}
\cvitem{データベース}{\techno{Amazon RDS}, \techno{PostgreSQL}, \techno{MySQL}}
\cvitem{モバイル}{\techno{Android}, \lang{Kotlin}}
\cvitem{通信プロトコル}{\techno{AMQP}, \techno{MQTT}, \techno{XMPP}}
\cvitem{プロジェクト管理}{\techno{Jira}, \techno{Redmine}}
\cvitem{CI/CD}{\techno{GitHub Actions}, \techno{Bitbucket Pipelines}, \techno{AWS CodeDeploy}, \techno{Jenkins}}
\cvitem{バージョン管理\newline{}システム}{\techno{Git}, \techno{Mercurial}, \techno{Subversion}}

\end{document}
