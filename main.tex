% !TEX program = xelatex

\documentclass{article}
% Force B5 PDF generation
\usepackage[b5paper]{geometry}
% Import from the submodule
\usepackage{rireki-style/rireki}
% Extra customizations for XeLaTeX
\usepackage{xltxtra}
\usepackage{fontspec}
\usepackage{xunicode}
\usepackage{xeCJK}
\usepackage{CJKulem}

% To support TeX conventions like "---"
\defaultfontfeatures{Mapping=tex-text,Scale=MatchLowercase}
\XeTeXlinebreaklocale "ja" % chktex 1 chktex 18
\setmainfont{AozoraMinchoRegular.ttf}
\setCJKmainfont{AozoraMinchoRegular.ttf}
\setCJKsansfont{AozoraMinchoLight.ttf}
\setCJKmonofont{AozoraMinchoLight.ttf}

% Redefine the date of today in Japanese, in order to display
% XXXX年XX月XX日現在on top of the first page
\renewcommand{\today}{\the\year{}年\the\month{}月\the\day{}日}

\begin{document}

%
% ID情報
%
\姓{Jeandeau}
\名{Alexis}
\姓読み{ジャンドー}
\名読み{アレクシ}
\性別{男}
\誕生日{平成2年7月2日}
\年齢{29}

\ifdefined\withphoto%
\顔写真{photo.jpg}
\else
\顔写真{}
\fi

%
% 現住所
%
\現住所郵便番号{164--0002}
\現住所{東京都中野区上高田 2--3--43}
\現住所読み{とうきょうと なかのく かみたかだ}
\現住所市外局番{}
\現住所電話番号{090--6791--5920}
\現住所呼び出し{}

%
% 連絡先
%
\連絡先郵便番号{}
\連絡先{\texttt{alexis.jeandeau@gmail.com}}
\連絡先読み{}
\連絡先市外局番{}
\連絡先電話番号{}
\連絡先呼び出し{}

%
% 学歴、職歴
%
% 学歴、職歴を年月順に列挙してください。合計20個まで記入出来ます。
% 20個を超える部分は印刷されませんので、ご注意ください。
% 印刷順は、学歴=>職歴の順になります。
%
%\学歴{平成1}{4}{○○市立◯◯高等学校 入学}      % {年}{月}{内容}
%\学歴{2001}{6}{École primaire d'Arc-sur-Tille 小学校 卒業}
\学歴{2005}{6}{Collège Champollion (中学校) 卒業}
\学歴{2005}{9}{Lycée Carnot (高等学校) 入学}
\学歴{2006}{9}{Lycée le Castel (高等学校) 編入}
\学歴{2008}{6}{Lycée le Castel (高等学校) 卒業}
\学歴{2008}{9}{IUT de Dijon (国立、コンピューター科学) 専門学校 入学}
\職歴{2010}{4}{CEA Valduc 3ヶ月のインターンシップ}
\学歴{2010}{6}{IUT de Dijon (国立、コンピューター科学) 専門学校 卒業}
\学歴{2011}{10}{ヨシダ日本語学院 入学}
\学歴{2012}{6}{ヨシダ日本語学院 卒業}
%\職歴{平成9}{4}{株式会社◯◯ 入社}
%\職歴{平成10}{9}{株式会社◯◯ 退職}
\職歴{2012}{6}{タカハシ技研合同会社 入社}
\職歴{2016}{4}{タカハシ技研合同会社 退職}
\職歴{2016}{4}{株式会社Z-Works 入社}
%\職歴{平成18}{10}{現在無職}
\職歴{}{}{現在に至る}

%
% 資格
%
% 資格を取得年月順に列挙してください。9つまで記入できます。
% 9つを超える部分は印刷されませんので、ご注意ください。
%
\資格{2008}{7}{普通運転免許(フランス) 取得}
\資格{2012}{7}{日本語能力試験2級 取得}
\資格{2014}{5}{普通運転免許(日本) 取得}

%
% 個人情報
%
% 志望の動機と本人希望記入欄はlatex のコマンドを記述できます。
%
\志望の動機{
    % \begin{tabular}{ll}
    %     { 志望の動機} & ∞∞∞∞∞∞ \\
    %     { 特技} & フランス語、英語 \\
    %     { アピールポイント} & ∞∞∞∞∞∞ \\
    % \end{tabular}
    現在Erlangを使ってIoTプラットフォームのサーバサイド開発に携わっていますが、
    ゲーム開発、特にオンライン \newline ゲームについてはずっと興味をもっています。 \\
    貴社では常に新しい技術を採用して大勢の人に利用されているゲームを作っている点に魅力を感じております。 \\
    現在Elixirを勉強中であり、貴社では本言語を使用したソフトウェアを開発しているとお聞きしました。 \\
    私も是非自分の知識や経験を活かしながらその一員として開発に参加できればと考えております。
}
\本人希望記入欄{
    % 私が希望する仕事の条件は下記の通りです。
    \begin{itemize}
        \item サーバサイドエンジニアを希望いたします。
        \item 現在就業中のため、平日10~19時はお電話に出られない場合があります。 \\
            折り返しご連絡いたしますので、恐れ入りますが留守番電話にメッセージを残していただけると幸いです。
        % \item 貴社の規定に従います
    \end{itemize}
}

%
% その他
%
\通勤時間{約 40分}
\扶養家族数{0}          % 人数(配偶者を除きます)
\配偶者{なし}           % あり|なし
\配偶者の扶養義務{なし} % あり|なし

% \サイン{A. Jeandeau}
\サイン{}

\end{document}
