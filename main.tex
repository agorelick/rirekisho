% !TEX program = lualatex

% See rireki-style/rireki.tex
\newif\iflualatex
\lualatextrue
\iflualatex
\documentclass[b5j]{ltjsarticle}
\usepackage[deluxe,nfssonly]{luatexja-preset}
\usepackage{graphicx}
\else
\documentclass[uplatex,b5j]{jsarticle}
\usepackage[dvipdfmx]{graphicx}
\fi
% Import from the submodule
\usepackage{rireki-style/rireki}

\空行挿入 % 学歴と職歴の間に空行を挿入します

\begin{document}

%
% ID情報
%
\姓{Jeandeau}
\名{Alexis}
\姓読み{ジャンドー}
\名読み{アレクシ}
\性別{男}
\誕生日{2020年7月2日}
\現在日付{\the\year{}年\the\month{}月\the\day{}日}
\年齢{30}

\ifdefined\withphoto%
\顔写真{photo.jpg}
\else
\顔写真{}
\fi

%
% 現住所
%
\現住所郵便番号{164--0002}
\現住所{東京都中野区上高田 2--3--43}
\現住所読み{とうきょうと なかのく かみたかだ}
\現住所市外局番{}
\現住所電話番号{090--6791--5920}
\現住所呼び出し{}

%
% 連絡先
%
\連絡先郵便番号{}
\連絡先{\texttt{alexis.jeandeau@gmail.com}}
\連絡先読み{}
\連絡先市外局番{}
\連絡先電話番号{}
\連絡先呼び出し{}

%
% 学歴、職歴
%
% 学歴、職歴を年月順に列挙してください。合計20個まで記入出来ます。
% 20個を超える部分は印刷されませんので、ご注意ください。
% 印刷順は、学歴=>職歴の順になります。
%
%\学歴{平成1}{4}{○○市立◯◯高等学校 入学}      % {年}{月}{内容}
%\学歴{2001}{6}{École primaire d'Arc-sur-Tille 小学校 卒業}
\学歴{2005}{6}{Collège Champollion (中学校) 卒業}
\学歴{2005}{9}{Lycée Carnot (高等学校) 入学}
\学歴{2006}{9}{Lycée le Castel (高等学校) 編入}
\学歴{2008}{6}{Lycée le Castel (高等学校) 卒業}
\学歴{2008}{9}{IUT de Dijon (国立、コンピューター科学) 専門学校 入学}
\職歴{2010}{4}{CEA Valduc 3ヶ月のインターンシップ}
\学歴{2010}{6}{IUT de Dijon (国立、コンピューター科学) 専門学校 卒業}
\学歴{2011}{10}{ヨシダ日本語学院 入学}
\学歴{2012}{6}{ヨシダ日本語学院 卒業}
%\職歴{平成9}{4}{株式会社◯◯ 入社}
%\職歴{平成10}{9}{株式会社◯◯ 退職}
\職歴{2012}{6}{タカハシ技研合同会社 入社}
\職歴{2016}{4}{タカハシ技研合同会社 退職}
\職歴{2016}{4}{株式会社Z-Works 入社}
%\職歴{平成18}{10}{現在無職}
\職歴{}{}{現在に至る}

%
% 資格
%
% 資格を取得年月順に列挙してください。9つまで記入できます。
% 9つを超える部分は印刷されませんので、ご注意ください。
%
\資格{2008}{7}{普通運転免許(フランス) 取得}
\資格{2012}{7}{日本語能力試験2級 取得}
\資格{2014}{5}{普通運転免許(日本) 取得}

%
% 個人情報
%
% 志望の動機と本人希望記入欄はlatex のコマンドを記述できます。
%
\志望の動機{
    別紙 職務履歴書をご参照ください
}
\本人希望記入欄{
}

%
% その他
%
\通勤時間{約 40分}
\扶養家族数{0}
\配偶者{なし}
\配偶者の扶養義務{なし}

% \サイン{A. Jeandeau}
\サイン{}

\end{document}
